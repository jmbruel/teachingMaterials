% Options:
% withid: adds a header line for writing the student's ID
% We recommand to keep the twoside option as it permits to add a specific header (student' name) on each new page.
\documentclass[a4paper,twoside,withid]{exam}

% packages already loaded: graphicx, geometry, listings, lastpage, fancyhdr, xcolor, mathpazo, ifthen, xstring, qcm, enumitem, amssymb, etoolbox, multido

\usepackage[final,activate={true,nocompatibility},kerning=true,spacing=true]{microtype}
\usepackage[english,french]{babel}
\usepackage[scaled=0.9]{DejaVuSansMono}
\usepackage{lipsum} % For demo. To remove
\usepackage{multicol}

% You can adjust the line spread of the font used in this document (mathpazo).
%\linespread{1.05}

% You can override the default page dimensions, mainly left and right (touching the other parameters may break the template):
%\geometry{bottom=0.5cm, top=-1cm, left=1.3cm, right=1.3cm, headheight=5cm, includefoot, includehead}

% You can customise the centered footpage:
%\cfoot[\large\thepage~~/~~\pageref{LastPage}]{\large\thepage~~/~~\pageref{LastPage}}
% but do not change the \lhead, \rhead, \lfoot, \rfoot, and \chead as they are used for page detection and IDs

\usepackage{courier} %% Sets font for listing as Courier.
\lstset{
tabsize = 4, %% set tab space width
showstringspaces = false, %% prevent space marking in strings, string is defined as the text that is generally printed directly to the console
%numbers = left, %% display line numbers on the left
commentstyle = \color{green}, %% set comment color
keywordstyle = \color{blue}, %% set keyword color
stringstyle = \color{red}, %% set string color
rulecolor = \color{black}, %% set frame color to avoid being affected by text color
basicstyle = \small \ttfamily , %% set listing font and size
breaklines = true, %% enable line breaking
numberstyle = \tiny,
}

\title{\vspace*{-1cm}\huge R2.01 -- Conception}
\author{2023-2024\\[0.1cm]
\textbf{Une seule page A4 recto/verso manuscrite autorisée.}\\[0.1cm]
Durée : 20' -- Sujet écrit sur \pageref{LastPage} pages
}
\date{5 avril 2024}


\begin{document}


% Creates the exam header
\begin{minipage}{\textwidth}
   \maketitle
\end{minipage}

% The template contains several exam commands:
% \exerciceExam{number}{title} creates a section dedicated to a given exercice. Replace 'number' by the number of points of the exam. 'title' is the title of the exercice (if no title, do not forget to put {})

% \questionExam{text} creates a numbered question. Replace 'text' by the title of the question.

% \begin{qcmExam}\end{qcmExam} creates a QCM environment


\noindent\fbox{\parbox{\textwidth}{%
Écrivez directement sur le sujet dans les zones de réponse. Ne pas déborder de ces zones.\\
\textbf{N'oubliez pas de mettre votre nom sur chaque page.}%
}}


\exerciceExam{3}{Diagramme de classe UML et code Java}

% \inlineAnswerBox as one argument: the length of the box (computed in number of characters). By default 10 characters.
\questionExam{Soit le diagramme de classe suivant. Indiquez, pour chaque classe, combien son implémentation Java aura d'attributs.

\includegraphics[width=0.3\textwidth]{dc1.png}

\begin{itemize}
	\item Xorg~~~~~~~~~~\inlineAnswerBox{}
	\item Schpounz \inlineAnswerBox{}
	\item Plunt~~~~~~~~~\inlineAnswerBox{}
\end{itemize}

}

\questionExam{Donnez un exemple de constructeur de la classe \textsf{Plunt} qui respecte le diagramme de classe de la question précédente.}

% First argument: color (optional, by default black). The color of the dots. Put white for no dot.
% Second parameter: the line spacing
% Third argument: the number of lines
% Fourth argument: the number of columns
\answerExam[gray]{1.5}{8}{1}

%\newpage
%-------------------------------------------------
\exerciceExam{3}{Diagramme de classe}
%-------------------------------------------------

\questionExam{Réalisez un diagramme de classe correspondant au domaine suivant (avec les attributs et les méthodes pertinentes\footnote{Ne pas donner le code des méthodes.}):

Un artisan pêcheur souhaite développer une application permettant de gérer son entreprise de pêche. 
Son entreprise possède des bateaux, chacun étant identifié par son modèle, son nom, son numéro d’immatriculation et la date de mise en service. 
Le nombre minimum et maximum de personnes à bord, la quantité totale maximale de stock sont des données communes à tous les bateaux d'un même modèle. 
On ajoutera au diagramme la classe Main qui stocke dans ses attributs la liste des bateaux et des modèles.

L'artisan souhaite pouvoir obtenir facilement l'âge d'un bateau, et le tonnage maximum de sa flotte actuelle (sur l'ensemble de ses bateaux).
}

% First argument: color (optional, by default black). The color of the dots. Put white for no dot.
% Second parameter: the line spacing
% Third argument: the number of lines
% Fourth argument: the number of columns
\answerExam[gray]{1.5}{14}{1}


%-------------------------------------------------
\exerciceExam{4}{Modeles et code}
%-------------------------------------------------

\begin{multicols}{2}
\begin{qcmExam}
	\item Parmi les relations suivantes, lesquelles sont des relations possibles entre cas d'utilisation (plusieurs réponses possibles) ?
		
		\begin{qcmExam}
		\item $<< extend >>$
		\item $<< include >>$
		\item Association
		\item Dépendance 
		\item Héritage 
		\item Aucune des autres réponses
		\end{qcmExam}

\item La relation d'héritage qui relie deux classes :
		
	\begin{qcmExam}
	\item Ne peut pas avoir de nom
	\item Peut se lire ``est un"
	\item Peut se lire ``a un"
	\item Possède des multiplicités
	\end{qcmExam}

\item Une classe Java contient tous ses objets ?
	
	\begin{qcmExam}
	\item vrai
	\item faux
	\end{qcmExam}

\item La notion de "navigation" d'une association $A \rightarrow B$ représente le fait que :
		
	\begin{qcmExam}
	\item Les objets de la classe $A$ ont accès aux références d'objets de la classe $B$
	\item Les objets de la classe $B$ ont accès aux références d'objets de la classe $A$
	\item Les objets de la classe $B$ n'ont pas accès aux références d'objets de la classe $A$
	\item Les objets de la classe $A$ n'ont pas accès aux références d'objets de la classe $B$
	\end{qcmExam}
				
\end{qcmExam}
\end{multicols}

%-------------------------------------------------
\end{document}
